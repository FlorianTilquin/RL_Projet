%% \begin{mygraph}{xmin=0, xmax=1, %
%%                ymin=0, ymax=1, %
%%                sizex=2.5, sizey=2.5}%
%%                {nomx=Axe X, nomy=Axe Y}%
%%                {0,.5,1}{0,0.25,...,1.05}

%% \graduationX[dashed, blue]{ .78 / $\frac{\pi}{4}$ }{ PARAMETRE TEXT }

%% \begin{mylegend}{x=0.3, y=.9, n=2, t=2.1, scale=.5}
%%   \newlegend{blue}{Courbe 1}
%%   \newlegend{red}{Courbe 2}
%% \end{mylegend}

%% \fillbetweencurve[opacity=.2, blue]{ COURBE 1 }{ COURBE 2 }

%% \end{mygraph}


\usepackage{tikz}


\pgfkeys{
%
 /mygraph/.is family, /mygraph,
 xmin/.estore in = \xn,
 xmax/.estore in = \xm,
 ymin/.estore in = \yn,
 ymax/.estore in = \ym,
 sizex/.estore in = \xx,
 sizey/.estore in = \yy,
 %
/mygraphb/.is family, /mygraphb,
 nomx/.estore in = \axex,
 nomy/.estore in = \axey,
%
/mygraphc/.is family, /mygraphc,
 gradsize/.estore in = \gradsize,
 gradsize/.default = 0.1,
 nomydist/.estore in = \axeyd,
 nomydist/.default = 0.8cm,
 gradsize, nomydist,                 % NE PAS OUBLIER
%
/myleg/.is family, /myleg,
 x/.estore in = \legendx,
 y/.estore in = \legendy,
 n/.estore in = \legendn,
 t/.estore in = \legendt,
 scale/.estore in = \legends,
 scale/.default = 1,
 scale,
%
/mylego/.is family, /mylego,
size/.estore in = \legendwidth,
size/.default = 0.4,
size                                  % NE PAS OUBLIER
}


%%%%%%%%%%%%%%%%%%%%%%%%%%%%%%%%%%%%%%%%%%%%%%%%%%%%%%%%%%%%%%%%%%%%%%%%%%%%%%%%
%                                                                              %
%%%%%%%%%%%%%%%%%%%%%%%%%%%%%%%%%%%%%%%%%%%%%%%%%%%%%%%%%%%%%%%%%%%%%%%%%%%%%%%%

\newenvironment{mygraph}[5][]{%
\pgfkeys{/mygraph, #2}
\pgfkeys{/mygraphb, #3}
\pgfkeys{/mygraphc, #1}
  \pgfmathsetmacro\dum{\yy/(\ym-\yn)}
  \pgfmathsetmacro\dumm{\xx/(\xm-\xn)}
\begin{tikzpicture}[yscale=\dum, xscale=\dumm,font=\sffamily]
  \pgfmathsetmacro\gradx{\gradsize / \dum}
  \pgfmathsetmacro\grady{\gradsize / \dumm}

  \foreach \x in {#4}{
    \draw[very thin, color=black, dotted] (\x,\yn) -- (\x,\ym);
    \draw (\x,\yn+\gradx) -- (\x,\yn)
          node[font=\tiny, anchor=north] {\pgfmathprintnumber{\x}};
  };
  \foreach \y in {#5}{
    \draw[very thin, color=black, dotted] (\xn,\y) -- (\xm,\y); 
    \draw (\xn+\grady,\y) -- (\xn,\y)
          node[font=\tiny, anchor=east] {\pgfmathprintnumber{\y}};
  };
  \draw (\xn,\yn) -- node[font=\scriptsize, below=0.3cm] {\axex} (\xm,\yn);
  \draw (\xn,\yn) -- node[font=\scriptsize, rotate=90, above=\axeyd, anchor=mid] {\axey} (\xn,\ym);
  \draw (\xn,\ym) -- (\xm,\ym);
  \draw (\xm,\yn) -- (\xm,\ym);

  \begin{scope}
    \clip (\xn,\yn) rectangle (\xm,\ym);
    %% \draw[dashed] (\xn, 0) -- (\xm, 0);
    %% \draw[dashed] (0, \yn) -- (0, \ym);
}{
  \end{scope}
\end{tikzpicture}
}

%%%%%%%%%%%%%%%%%%%%%%%%%%%%%%%%%%%%%%%%%%%%%%%%%%%%%%%%%%%%%%%%%%%%%%%%%%%%%%%%

\newenvironment{mylegend}[2][]{
\pgfkeys{/myleg, #2}
\pgfkeys{/mylego, #1}
\begin{scope}[shift={(\legendx,\legendy)}, scale=\legends]
\pgfmathsetmacro\legendwidth{\legendwidth * (\xm-\xn) / \xx }
\pgfmathsetmacro\dum{ (0.125 * (\ym-\yn) / \yy) }
\pgfmathsetmacro\dumm{ - (0.125 * (\xm-\xn) / \xx) }
\pgfmathsetmacro\legendy{ 0 }
\coordinate (dum) at (\dumm,\dum);
\pgfmathsetmacro\dum{ - (\legendn-0.4)*(0.25 * (\ym-\yn) / \yy) }
\coordinate (dumm) at (\dumm,\dum);
\pgfmathsetmacro\dumm{\dumm + (\legendt * (\xm-\xn) / \xx) }
\draw[fill=white, opacity=.8] (dum) -- (dumm) -| (\dumm,\dum) |- (dum);
}{
\end{scope}
%% \pgfmathsetmacro\legendyi{\legendyi + (0.125 * (\ym-\yn) / \yy)  }
%% \pgfmathsetmacro\legendy{\legendy + (0.1 * (\ym-\yn) / \yy)  }
%% \pgfmathsetmacro\legendx{\legendx - (0.125 * (\xm-\xn) / \xx)  }
%% \draw[] (\legendx,\legendyi) -- (\legendx,\legendy) %
%%                              -| (\legendx + 1,\legendy)
%%                              |- (\legendx,\legendyi);
}

\newcommand{\newlegend}[2]{
\draw[font=\scriptsize, #1] (0,\legendy) -- (\legendwidth,\legendy)	node[right,scale=\legends]{#2};
\pgfmathsetmacro\legendy{\legendy - (0.25 * (\ym-\yn) / \yy)  }
}

%%%%%%%%%%%%%%%%%%%%%%%%%%%%%%%%%%%%%%%%%%%%%%%%%%%%%%%%%%%%%%%%%%%%%%%%%%%%%%%%

\newenvironment{outofbox}{%
  \end{scope}%
}{%
  \begin{scope}%
    \clip (\xn,\yn) rectangle (\xm,\ym);%
}

%%%%%%%%%%%%%%%%%%%%%%%%%%%%%%%%%%%%%%%%%%%%%%%%%%%%%%%%%%%%%%%%%%%%%%%%%%%%%%%%

\newcommand{\fillbetweencurve}[3][]{
\begin{scope}
\clip (\xn,\yn) -- #2 -- (\xm,\yn) -- cycle;
\fill[#1] (\xn,\ym) -- #3 -- (\xm,\ym) -- cycle;
\end{scope}
}

%%%%%%%%%%%%%%%%%%%%%%%%%%%%%%%%%%%%%%%%%%%%%%%%%%%%%%%%%%%%%%%%%%%%%%%%%%%%%%%%

\newcommand{\graduationX}[3][very thin, color=black, dotted]{
\end{scope}
  \foreach \x/\t in {#2}{
    \draw[#1] (\x,\yn) -- (\x,\ym);
    \draw (\x,\yn+\gradx) -- (\x,\yn)
          node[font=\tiny, anchor=north, #3] {\t};
  };
\begin{scope}%
\clip (\xn,\yn) rectangle (\xm,\ym);%
}

%%%%%%%%%%%%%%%%%%%%%%%%%%%%%%%%%%%%%%%%%%%%%%%%%%%%%%%%%%%%%%%%%%%%%%%%%%%%%%%%

\newcommand{\graduationY}[3][very thin, color=black, dotted]{
\end{scope}
  \foreach \y/\t in {#2}{
    \draw[#1] (\xn,\y) -- (\xm,\y);
    \draw (\xn+\grady,\y) -- (\xn,\y)
          node[anchor=east, font=\tiny, shift={(-0*\grady,0)}, #3] {\t};
  };
\begin{scope}%
\clip (\xn,\yn) rectangle (\xm,\ym);%
}
